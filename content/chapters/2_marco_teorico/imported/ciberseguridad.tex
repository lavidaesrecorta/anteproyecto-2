\section{Ciberseguridad}

\subsection{Confidencialidad, Integridad y Disponibilidad}
\hfil \\
La protección de la información privada es un desafío que antecede
a la era digital, pero ha alcanzado nuevas dimensiones y complejidades gracias
a diversos avances tecnológicos y la masiva interconexión global de sistemas
informáticos \cite{greenbergCrashOverrideMalware2017,ibmCostDataBreach2023,jinUnderstandingIoTSecurity2022}. La facilidad del acceso a la información a través del ciberespacio
ha exigido la creación de la disciplina de la ciberseguridad. La ciberseguridad
es el conjunto de prácticas y herramientas que permiten resguardar la seguridad
de un sistema informático y su objetivo es garantizar la confidencialidad,
integridad y disponibilidad de la información \cite{jironAPLICACIONESREDESNEURONALES2023}, que se describen a continuación:
\begin{itemize}
    \item Confidencialidad se refiere a los esfuerzos realizados para asegurar que los datos permanezcan secretos o privados de manera que la información solamente pueda ser accedida por usuarios autorizados.
    \item Integridad implica asegurar que los datos sean auténticos, precisos y confiables, y que no hayan sido alterados de manera indebida.
    \item Disponibilidad significa que las personas autorizadas deben poder acceder a la información cuando la necesiten, de manera oportuna y sin retrasos excesivos. % \cite{fortinet-cia-triad}
\end{itemize}




\subsection{Protocolos para una Comunicación Segura}
\hfil \\
Un protocolo de internet contiene los lineamientos que establecen el formato
y la secuencia de acciones necesarias para establecer una conexión exitosa
entre dos o más dispositivos. Estos lineamientos garantizan la coherencia de la comunicación digital y han dado forma a la transmisión a través del internet \cite{schweitzer-2002}.
Existen diversos protocolos diseñados para distintos tipos de
aplicaciones, que incluyen la transferencia de archivos, túneles seguros para la administración remota y protocolos de autenticación. Para lograr una comunicación segura, diversos protocolos
han implementado algoritmos de encriptación compuestos, es decir, protocolos de encriptación que utilizan tanto criptografía
simétrica como asimétrica. Por ejemplo, el protocolo de acceso remoto Secure Shell (SSH) utiliza DHKE para el intercambio de llaves y Advanced Encryption Standard (AES) para el cifrado de la sesión remota \cite{paar-2011}.

\subsection{Control de Acceso y Autenticación}
\hfil \\
Los sistemas informáticos mantienen sistemas de control de acceso que
asocian identidades con niveles de autorización que regulan el privilegio de
revisión y modificación del contenido de los repositorios presentes en el
sistema. Sin embargo, también necesitan comprobar la legitimidad de la
identidad que presenta un usuario que intenta acceder a la información. Un
proceso de autenticación verifica la identidad que un usuario utiliza en el
proceso de identificación. 
La deficiente o nula implementación de sistemas de control de acceso ha sido identificado como uno de los problemas
cruciales a la hora de analizar la ciberseguridad de dispositivos en la Internet de las Cosas (Internet of Things, IoT), en donde
muchos dispositivos son desplegados con protocolos de autenticación
vulnerables, débiles o inexistentes \cite{jinUnderstandingIoTSecurity2022}.

