\section{Internet de las Cosas (IoT)}
\subsection{Aplicaciones y Usos del Internet de las Cosas}
\hfil \\
El internet de las cosas (IoT, por sus siglas en inglés) es un término que
alude a una red de dispositivos físicos, vehículos, electrodomésticos y otros
equipos electrónicos que cuentan con sensores y la capacidad de conectarse a
una red que le permite recopilar y transmitir datos.
El internet de las cosas permite a sistemas inteligentes crear una vasta
red de dispositivos interconectados que pueden ejecutar tareas de manera
autónoma \cite{ibmCostDataBreach2023}. 

Actualmente, el impacto del internet de las cosas en la industria y el
sector doméstico puede ser percibido en su rápido crecimiento e
implementación. Se espera que el crecimiento de los dispositivos desplegados
mantenga un ritmo exponencial en los años venideros \cite{meolaLookExamplesIoT2023}. Diversas industrias han adoptado 
el uso de dispositivos IoT para tareas de monitoreo en sistemas de telemedicina, redes de comunicación
celular (3G/4G/LTE/5G), redes de sensores a larga escala, redes eléctricas inteligentes, ciudades inteligentes, entre otros \cite{mumtazGuestEditorial5G2018}.


\subsection{Internet de las Cosas y su Impacto en la Ciberseguridad}
\hfil \\
El uso de dispositivos de bajo costo para tareas de monitoreo a través de Internet ha permitido aumentar la productividad, así como mejorar la reportabilidad y el control de procesos, incluso en empresas de menor escala. No obstante, la prioridad de mantener bajos costos de desarrollo para conservar la competitividad en el mercado de IoT, junto con procesos de diseño acotados, ha dado lugar a un mercado saturado de productos vulnerables, cuyo uso impacta de forma directa en la seguridad del consumidor. 
Por ejemplo, \cite{jinUnderstandingIoTSecurity2022} ha detectado que cerca del 90\% de aplicaciones móviles que acompañan soluciones IoT en Google Play, contienen al menos una falla de seguridad, siendo cerca del 44\% la falta de implementación de un protocolo de cifrado.

A pesar de su potencial, los sistemas embebidos que componen gran
parte del internet de las cosas suelen ser dispositivos orientados a usos
específicos y con bajos niveles de procesamiento, esto ha significado que
algunas técnicas y protocolos ampliamente aceptados en el campo de la
ciberseguridad no sean aplicables en los entornos IoT \cite{krishnasrijaLightweightMutualTransitive2023}.

Actores maliciosos son capaces de usar las vulnerabilidades de los dispositivos IoT para causar daño en el mundo real, generando
interrupciones de servicio, daño físico a los equipos vulnerados o incluso la muerte a pacientes y trabajadores conectados a equipos de misión crítica \cite{alsubaeiIoMTSAFInternetMedical2019,greenbergCrashOverrideMalware2017}. En el año 2017, un malware apodado \textit{Mirai} utilizó la falta de mecanismos de autenticación en dispositivos IoT para generar una red de computadores, la cual fué utilizada para ejecutar un ataque a la infraestructura de \textit{DynDNS}, lo que significó la interrupción de servicio de sus clientes, altos costos de recuperación y un impacto permanente a la reputación que generó la perdida de cerca del 8\% de su clientela. 