\chapter{Planteamiento del Problema}
\section{Descripción de la Problemática}
Actualmente la generación de claves es realizada mediante algoritmos que utilizan aritmética modular y factorización de números primos de gran tamaño, lo que puede significar un uso elevado de recursos computacionales. Para la regeneración de claves, los participantes deben realizar el proceso de intercambio completo, creando una alta carga computacional de la misma manera que el proceso de intercambio de claves inicial. En un servidor web, las operaciones utilizadas por algoritmos como RSA en TLS corresponden entre un 20-58\% del tiempo de ejecución \cite{coarfa-2006}. 

El uso de Tree Parity Machines (TPM) en la criptografía neuronal emerge como una alternativa viable, debido a que utiliza operaciones 
matemáticas simples y no requiere del uso de grandes números enteros \cite{teodoroFPGABasedPerformanceEvaluation2021}.

Este trabajo propone un estudio sistemático y experimental sobre variaciones multicapa de las arquitecturas TPM, denominadas Multilayer Tree Parity Machuine (MTPM), las cuales permiten explorar nuevas configuraciones topológicas que podrían ofrecer mejoras en eficiencia en el proceso de sincronización. Para ello, se ha desarrollado un motor de simulación que automatiza el proceso de sincronización entre redes MTPM, permitiendo el análisis comparativo de múltiples arquitecturas y la influencia de los parámetros $H$, $K$, $N$, $L$ y $M$. Además, se ha construido una interfaz gráfica que facilita la visualización de los resultados a través de gráficos interactivos y representaciones tridimensionales.


\section{Originalidad de la Investigación}
La originalidad de esta investigación radica en la combinación de estímulos no binarios y la presencia de múltiples capas ocultas en la arquitectura de una red Tree Parity Machine. 

Finalmente, basado en la revisión bibliográfica, se elegirá el ataque que ha probado ser efectivo contra el proceso de sincronización neuronal, y a continuación se aplicará sobre el protocolo propuesto en esta investigación.
\section{Pregunta de Investigación}
¿Cuáles son los efectos del traslape en los estímulos de
entrada de cada neurona en una capa oculta, en el proceso de
sincronización neuronal de dos redes Multilayer Tree Parity
Machine?
\section{Hipótesis}
El uso estímulos sin traslape en la entrada en cada neurona de
una capa oculta, permite que en el proceso de sincronización
neuronal entre dos redes Multilayer Tree Parity Machine termine
correctamente.
